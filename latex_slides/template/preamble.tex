\usepackage{helvet}
\usepackage[english]{babel}
\usepackage{pgfplots}
\usepackage{pgf}
\pgfplotsset{compat=newest}
\usepackage{booktabs}
\usepackage[T1]{fontenc}
\usepackage[utf8]{inputenc}
\usepackage{lipsum}
\usepackage{tcolorbox}
\usepackage{xcolor}
\usepackage{pifont}   % for \ding symbols
\usepackage{listings}
\usepackage{microtype}
\usepackage{float}
%\usepackage{gensymb}
\usepackage{lmodern}
\usepackage{bm}
\usepackage{siunitx}
\usepackage{multicol}
\usepackage{media9}
\usepackage{animate}
\usepackage{longtable}
\usepackage{pgffor}
\usepackage{hyperref}
%\usepackage{enumitem}
\usepackage{dsfont}
\usepackage{caption}
\usepackage{subcaption}
\usepackage[version=4]{mhchem}
\usepackage{chemplants}
\usepackage{tikz}
\usetikzlibrary{positioning,arrows.meta, calc}
\usepackage{nomencl} % Include the nomenclature package
\usepackage{csquotes}       % For biblatex with babel
\usepackage[backend=biber,sorting = none]{biblatex} % Package for bibliography (citing)
\bibliography{bibliography.bib}

\definecolor{limegreen}{RGB}{153, 255, 51}

% Nomenclature
\makenomenclature % Create the command to produce the nomenclature
\renewcommand{\nomgroup}[1]{%
  \ifthenelse{\equal{#1}{A}}{\item[\textbf{Variables}]}{%
  \ifthenelse{\equal{#1}{B}}{\item[\textbf{Greek letters}]}{%
  \ifthenelse{\equal{#1}{C}}{\item[\textbf{Sub- and superscripts}]}{%
  \ifthenelse{\equal{#1}{D}}{\item[\textbf{QuickFix}]}}}}}


\input{template/settings}

\newcommand{\setcolor}[1]{\def\chosencolor{#1}}
\newcommand{\setdepartment}[1]{\def\department{#1}}

\newcommand*\circled[1]{\tikz[baseline=(char.base)]{
            \node[shape=circle,inner sep=0.6pt,fill = blue] (char) {\color{white}\scriptsize #1};}}

\usetheme{DTU}
\setbeamersize{text margin left=5mm}
\def\insertframetitle{}

\newcommand{\inserttitlepage}{
    \begin{frame}[plain, noframenumbering]{}

        \begin{center}
        \hspace{-3.6em}
            \includegraphics[width = 0.25\paperwidth]{logos/\targetcolourmodel/white.pdf}
        \end{center}
        
    \end{frame}

    \begin{frame}[plain]{}
        \color{white}\maketitle    
    \end{frame}

    \setbeamercolor{background canvas}{bg = white}
}

\usepackage{listings}  			% Placer kildekode i dokumentet med 
% ¤¤ Opsaetning af listings ¤¤ %
\definecolor{commentGreen}{RGB}{34,139,24}
\definecolor{stringPurple}{RGB}{208,76,239}

\lstset{language=Matlab,					% Sprog
	basicstyle=\ttfamily\tiny,		% Opsaetning af teksten
	keywords={for,if,while,else,elseif,		% Noegleord at fremhaeve
			  end,break,return,case,
			  switch,function},
	keywordstyle=\color{blue},				% Opsaetning af noegleord
	commentstyle=\color{commentGreen},		% Opsaetning af kommentarer
	stringstyle=\color{stringPurple},		% Opsaetning af strenge
	showstringspaces=false,					% Mellemrum i strenge enten vist eller blanke
	numbers=left, numberstyle=\tiny,		% Linjenumre
	extendedchars=true, 					% Tillader specielle karakterer
	columns=flexible,						% Kolonnejustering
	breaklines, breakatwhitespace=true,		% Bryd lange linjer
	literate=														%Muliggør æ,ø,å
	{á}{{\'a}}1 {é}{{\'e}}1 {í}{{\'i}}1 {ó}{{\'o}}1 {ú}{{\'u}}1
	{Á}{{\'A}}1 {É}{{\'E}}1 {Í}{{\'I}}1 {Ó}{{\'O}}1 {Ú}{{\'U}}1
	{à}{{\`a}}1 {è}{{\`e}}1 {ì}{{\`i}}1 {ò}{{\`o}}1 {ù}{{\`u}}1
	{À}{{\`A}}1 {È}{{\'E}}1 {Ì}{{\`I}}1 {Ò}{{\`O}}1 {Ù}{{\`U}}1
	{ä}{{\"a}}1 {ë}{{\"e}}1 {ï}{{\"i}}1 {ö}{{\"o}}1 {ü}{{\"u}}1
	{Ä}{{\"A}}1 {Ë}{{\"E}}1 {Ï}{{\"I}}1 {Ö}{{\"O}}1 {Ü}{{\"U}}1
	{â}{{\^a}}1 {ê}{{\^e}}1 {î}{{\^i}}1 {ô}{{\^o}}1 {û}{{\^u}}1
	{Â}{{\^A}}1 {Ê}{{\^E}}1 {Î}{{\^I}}1 {Ô}{{\^O}}1 {Û}{{\^U}}1
	{œ}{{\oe}}1 {Œ}{{\OE}}1 {æ}{{\ae}}1 {Æ}{{\AE}}1 {ß}{{\ss}}1
	{ű}{{\H{u}}}1 {Ű}{{\H{U}}}1 {ő}{{\H{o}}}1 {Ő}{{\H{O}}}1
	{ç}{{\c c}}1 {Ç}{{\c C}}1 {ø}{{\o}}1 {å}{{\r a}}1 {Å}{{\r A}}1
	{€}{{\euro}}1 {£}{{\pounds}}1 {«}{{\guillemotleft}}1
	{»}{{\guillemotright}}1 {ñ}{{\~n}}1 {Ñ}{{\~N}}1 {¿}{{?`}}1,
	backgroundcolor=\color{white},	%Baggrundsfarve
	captionpos=t
}
\renewcommand{\lstlistingname}{Matlab code}% Listing -> Matlab code


\AtBeginSection[]{
  \begin{frame}{Outline}
    \tableofcontents[currentsection]
  \end{frame}
}

% \hypersetup{
%   colorlinks=true,
%   linkcolor=blue,
%   urlcolor=blue
% }

% Colors
 \definecolor{LimeGreen}{RGB}{38, 255, 75} 


\newcommand{\deriv}[2]{\frac{\partial #1}{\partial #2}}

\newcommand{\sectionnew}[1]{\section{#1} \frame{\tableofcontents[currentsection]}}

\newcommand{\onlybm}[3]{\only<#2>{\boldsymbol{#1}}\only<#3>{#1}}

\newcommand{\pcite}[1]{(\cite{#1})}

% Coloring box for outputs/states/inputs
\newcommand{\mycolorbox}[4]{%
    % #1 = text
    % #2 = color
    % #3 = x coordinate
    % #4 = y coordinate
    \node[rectangle, 
        fill=#2!50, 
        opacity=0.5, 
        minimum width=6.5cm, 
        minimum height=1.8cm, 
        align=center
    ] at (#3,#4) {}; % background rectangle

    \node[rectangle, 
        fill=#2, 
        opacity=1, 
        text opacity=1, 
        align=center
    ] at (#3,#4) {#1}; % foreground text
}



\newcommand{\showTestPlotTrainNineTen}[4]{
\begin{frame}{Model selection}
\footnotesize{
\begin{table}
 \centering
    \caption{Model evaluation and model test. Training data: Dataset 9-10}
\begin{tikzpicture}
    \node[inner sep=0pt] (tab) {\input{tables/RMSE_train_9_10.txt}};
    \node[rectangle,
    draw,
    very thick,
    minimum width = 1cm, 
    minimum height = 0.35 cm] at (#3,#4) {};    
\end{tikzpicture}
\end{table}
}
\end{frame}

\begin{frame}{Model selection}
\begin{figure}
    \centering
    \includegraphics[width=0.7\linewidth]{figures/results/full_hor_model_#1_train_9_10_test_#2.pdf}
    \caption{Full  horizon prediction. Model #1, test dataset #2, training dataset 9-10}
\end{figure}
\end{frame}

\begin{frame}{Model selection}
\footnotesize{
\begin{table}
 \centering
    \caption{Model evaluation and model test. Training data: Dataset 9-10}
\begin{tikzpicture}
    \node[inner sep=0pt] (tab) {\input{tables/RMSE_train_9_10.txt}};
    \node[rectangle,
    draw,
    very thick,
    minimum width = 1cm, 
    minimum height = 0.35 cm] at (#3,#4) {};    
\end{tikzpicture}
\end{table}
}
\end{frame}

}






\newcommand{\showTestPlotTrainElevenThirteen}[4]{
\begin{frame}{Model selection}
\footnotesize{
\begin{table}
 \centering
    \caption{Model evaluation and model test. Training data: Dataset 11-13}
\begin{tikzpicture}
    \node[inner sep=0pt] (tab) {\input{tables/RMSE_train_11_12_13.txt}};
    \node[rectangle,
    draw,
    very thick,
    minimum width = 1cm, 
    minimum height = 0.35 cm] at (#3,#4) {};    
\end{tikzpicture}
\end{table}
}
\end{frame}

\begin{frame}{Model selection}
\begin{figure}
    \centering
    \includegraphics[width=0.7\linewidth]{figures/results/full_hor_model_#1_train_11_12_13_test_#2.pdf}
    \caption{Full  horizon prediction. Model #1, test dataset #2, training dataset 11-13}
\end{figure}
\end{frame}

\begin{frame}{Model selection}
\footnotesize{
\begin{table}
 \centering
    \caption{Model evaluation and model test. Training data: Dataset 11-13}
\begin{tikzpicture}
    \node[inner sep=0pt] (tab) {\input{tables/RMSE_train_11_12_13.txt}};
    \node[rectangle,
    draw,
    very thick,
    minimum width = 1cm, 
    minimum height = 0.35 cm] at (#3,#4) {};    
\end{tikzpicture}
\end{table}
}
\end{frame}

}

\newcommand{\inserttables}[3]{%
  \foreach \i in {1,...,#2} {
    \begin{frame}{#3}
    \scriptsize{
      \begin{table}[]
        \centering
        \input{tables/par_est_\i_#1.txt}
      \end{table}
    }
    \end{frame}
  }
  \begin{frame}{#3}
  \scriptsize{
    \begin{table}[]
      \centering
      \input{tables/RMSE_#1.txt}
    \end{table}
  }
  \end{frame}
}

\newcommand{\inserttablesnew}[4]{%
  \foreach \i in {1,...,#2} {
    \begin{frame}{#4}
    \scriptsize{
    Parameter estimate
      \begin{table}[]
        \centering
        \input{tables/par_est_\i_#1.txt}
      \end{table}
    }
    \end{frame}
  }
  \foreach \i in {1,...,#3} {
      \begin{frame}{#4}
      \scriptsize{  
      RMSE
        \begin{table}[]
          \centering
          \input{tables/RMSE_#1_\i.txt}
        \end{table}
      }
      \end{frame}
  }
}


\newcommand{\drawsprinkler}[3]{
    % Base horizontal line
    \draw[] ([yshift=#2]#1) -- ([xshift=#312mm,yshift=#2]#1);

    % Central sprinkle group
    \draw[] ([xshift=#37mm,yshift=#2]#1) -- ([xshift=#37mm,yshift=-1mm+#2]#1);
    \draw[] ([xshift=#37mm,yshift=#2]#1) -- ([xshift=#36.5mm,yshift=-0.8mm+#2]#1);
    \draw[] ([xshift=#37mm,yshift=#2]#1) -- ([xshift=#37.5mm,yshift=-0.8mm+#2]#1);

    % First new sprinkle group
    \draw[] ([xshift=#34mm,yshift=#2]#1) -- ([xshift=#34mm,yshift=-1mm+#2]#1);
    \draw[] ([xshift=#34mm,yshift=#2]#1) -- ([xshift=#33.5mm,yshift=-0.8mm+#2]#1);
    \draw[] ([xshift=#34mm,yshift=#2]#1) -- ([xshift=#34.5mm,yshift=-0.8mm+#2]#1);

    % Second new sprinkle group
    \draw[] ([xshift=#310mm,yshift=#2]#1) -- ([xshift=#310mm,yshift=-1mm+#2]#1);
    \draw[] ([xshift=#310mm,yshift=#2]#1) -- ([xshift=#39.5mm,yshift=-0.8mm+#2]#1);
    \draw[] ([xshift=#310mm,yshift=#2]#1) -- ([xshift=#310.5mm,yshift=-0.8mm+#2]#1);
}

\newcommand{\drawPumpTriangle}[2]{%
  \ifthenelse{\equal{#2}{right}}{%
    \draw[] (#1.0) -- (#1.125) -- (#1.235) -- cycle;%
  }{}%
  \ifthenelse{\equal{#2}{left}}{%
    \draw[] (#1.180) -- (#1.55) -- (#1.305) -- cycle;%
  }{}%
  \ifthenelse{\equal{#2}{up}}{%
    \draw[] (#1.90) -- (#1.215) -- (#1.325) -- cycle;%
  }{}%
  \ifthenelse{\equal{#2}{down}}{%
    \draw[] (#1.270) -- (#1.35) -- (#1.145) -- cycle;%
  }{}%
}


\newcommand{\kparams}[5]{%
  \textcolor{#1}{$k_{A,cga}$} \,
  \textcolor{#2}{$k_{A,cla}$} \,
  \textcolor{#3}{$k_{A,Ta}$} \,
  \textcolor{#4}{$k_{D,cld}$} \,
  \textcolor{#5}{$k_{D,Td}$}%
}



% Macro to insert metrics
\newcommand{\insertmetrics}[1]{%
  \csname metrics@#1\endcsname
}