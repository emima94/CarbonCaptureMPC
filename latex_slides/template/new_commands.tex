\newcommand{\deriv}[2]{\frac{\partial #1}{\partial #2}}

\newcommand{\sectionnew}[1]{\section{#1} \frame{\tableofcontents[currentsection]}}

\newcommand{\onlybm}[3]{\only<#2>{\boldsymbol{#1}}\only<#3>{#1}}

\newcommand{\pcite}[1]{(\cite{#1})}

% Coloring box for outputs/states/inputs
\newcommand{\mycolorbox}[4]{%
    % #1 = text
    % #2 = color
    % #3 = x coordinate
    % #4 = y coordinate
    \node[rectangle, 
        fill=#2!50, 
        opacity=0.5, 
        minimum width=6.5cm, 
        minimum height=1.8cm, 
        align=center
    ] at (#3,#4) {}; % background rectangle

    \node[rectangle, 
        fill=#2, 
        opacity=1, 
        text opacity=1, 
        align=center
    ] at (#3,#4) {#1}; % foreground text
}



\newcommand{\showTestPlotTrainNineTen}[4]{
\begin{frame}{Model selection}
\footnotesize{
\begin{table}
 \centering
    \caption{Model evaluation and model test. Training data: Dataset 9-10}
\begin{tikzpicture}
    \node[inner sep=0pt] (tab) {\input{tables/RMSE_train_9_10.txt}};
    \node[rectangle,
    draw,
    very thick,
    minimum width = 1cm, 
    minimum height = 0.35 cm] at (#3,#4) {};    
\end{tikzpicture}
\end{table}
}
\end{frame}

\begin{frame}{Model selection}
\begin{figure}
    \centering
    \includegraphics[width=0.7\linewidth]{figures/results/full_hor_model_#1_train_9_10_test_#2.pdf}
    \caption{Full  horizon prediction. Model #1, test dataset #2, training dataset 9-10}
\end{figure}
\end{frame}

\begin{frame}{Model selection}
\footnotesize{
\begin{table}
 \centering
    \caption{Model evaluation and model test. Training data: Dataset 9-10}
\begin{tikzpicture}
    \node[inner sep=0pt] (tab) {\input{tables/RMSE_train_9_10.txt}};
    \node[rectangle,
    draw,
    very thick,
    minimum width = 1cm, 
    minimum height = 0.35 cm] at (#3,#4) {};    
\end{tikzpicture}
\end{table}
}
\end{frame}

}






\newcommand{\showTestPlotTrainElevenThirteen}[4]{
\begin{frame}{Model selection}
\footnotesize{
\begin{table}
 \centering
    \caption{Model evaluation and model test. Training data: Dataset 11-13}
\begin{tikzpicture}
    \node[inner sep=0pt] (tab) {\input{tables/RMSE_train_11_12_13.txt}};
    \node[rectangle,
    draw,
    very thick,
    minimum width = 1cm, 
    minimum height = 0.35 cm] at (#3,#4) {};    
\end{tikzpicture}
\end{table}
}
\end{frame}

\begin{frame}{Model selection}
\begin{figure}
    \centering
    \includegraphics[width=0.7\linewidth]{figures/results/full_hor_model_#1_train_11_12_13_test_#2.pdf}
    \caption{Full  horizon prediction. Model #1, test dataset #2, training dataset 11-13}
\end{figure}
\end{frame}

\begin{frame}{Model selection}
\footnotesize{
\begin{table}
 \centering
    \caption{Model evaluation and model test. Training data: Dataset 11-13}
\begin{tikzpicture}
    \node[inner sep=0pt] (tab) {\input{tables/RMSE_train_11_12_13.txt}};
    \node[rectangle,
    draw,
    very thick,
    minimum width = 1cm, 
    minimum height = 0.35 cm] at (#3,#4) {};    
\end{tikzpicture}
\end{table}
}
\end{frame}

}

\newcommand{\inserttables}[3]{%
  \foreach \i in {1,...,#2} {
    \begin{frame}{#3}
    \scriptsize{
      \begin{table}[]
        \centering
        \input{tables/par_est_\i_#1.txt}
      \end{table}
    }
    \end{frame}
  }
  \begin{frame}{#3}
  \scriptsize{
    \begin{table}[]
      \centering
      \input{tables/RMSE_#1.txt}
    \end{table}
  }
  \end{frame}
}

\newcommand{\inserttablesnew}[4]{%
  \foreach \i in {1,...,#2} {
    \begin{frame}{#4}
    \scriptsize{
    Parameter estimate
      \begin{table}[]
        \centering
        \input{tables/par_est_\i_#1.txt}
      \end{table}
    }
    \end{frame}
  }
  \foreach \i in {1,...,#3} {
      \begin{frame}{#4}
      \scriptsize{  
      RMSE
        \begin{table}[]
          \centering
          \input{tables/RMSE_#1_\i.txt}
        \end{table}
      }
      \end{frame}
  }
}


\newcommand{\drawsprinkler}[3]{
    % Base horizontal line
    \draw[] ([yshift=#2]#1) -- ([xshift=#312mm,yshift=#2]#1);

    % Central sprinkle group
    \draw[] ([xshift=#37mm,yshift=#2]#1) -- ([xshift=#37mm,yshift=-1mm+#2]#1);
    \draw[] ([xshift=#37mm,yshift=#2]#1) -- ([xshift=#36.5mm,yshift=-0.8mm+#2]#1);
    \draw[] ([xshift=#37mm,yshift=#2]#1) -- ([xshift=#37.5mm,yshift=-0.8mm+#2]#1);

    % First new sprinkle group
    \draw[] ([xshift=#34mm,yshift=#2]#1) -- ([xshift=#34mm,yshift=-1mm+#2]#1);
    \draw[] ([xshift=#34mm,yshift=#2]#1) -- ([xshift=#33.5mm,yshift=-0.8mm+#2]#1);
    \draw[] ([xshift=#34mm,yshift=#2]#1) -- ([xshift=#34.5mm,yshift=-0.8mm+#2]#1);

    % Second new sprinkle group
    \draw[] ([xshift=#310mm,yshift=#2]#1) -- ([xshift=#310mm,yshift=-1mm+#2]#1);
    \draw[] ([xshift=#310mm,yshift=#2]#1) -- ([xshift=#39.5mm,yshift=-0.8mm+#2]#1);
    \draw[] ([xshift=#310mm,yshift=#2]#1) -- ([xshift=#310.5mm,yshift=-0.8mm+#2]#1);
}

\newcommand{\drawPumpTriangle}[2]{%
  \ifthenelse{\equal{#2}{right}}{%
    \draw[] (#1.0) -- (#1.125) -- (#1.235) -- cycle;%
  }{}%
  \ifthenelse{\equal{#2}{left}}{%
    \draw[] (#1.180) -- (#1.55) -- (#1.305) -- cycle;%
  }{}%
  \ifthenelse{\equal{#2}{up}}{%
    \draw[] (#1.90) -- (#1.215) -- (#1.325) -- cycle;%
  }{}%
  \ifthenelse{\equal{#2}{down}}{%
    \draw[] (#1.270) -- (#1.35) -- (#1.145) -- cycle;%
  }{}%
}


\newcommand{\kparams}[5]{%
  \textcolor{#1}{$k_{A,cga}$} \,
  \textcolor{#2}{$k_{A,cla}$} \,
  \textcolor{#3}{$k_{A,Ta}$} \,
  \textcolor{#4}{$k_{D,cld}$} \,
  \textcolor{#5}{$k_{D,Td}$}%
}



% Macro to insert metrics
\newcommand{\insertmetrics}[1]{%
  \csname metrics@#1\endcsname
}